\documentclass[aps,prl,preprint]{revtex4-2}
\usepackage{amsmath}
\usepackage{physics}

\begin{document}

\title{Title of Your Paper}
\author{Your Name}
\affiliation{Your Institution}
\date{\today}

\begin{abstract}
    Your abstract goes here.
\end{abstract}

% \maketitle
\section{Mean-field theory}
The mean-field Nambu-Gor'kov Hamiltonian is
\begin{align}
    H
     & = \frac{1}{2}\iint\dd[3]\vb{r}\dd[3]\vb{r}'
    \widehat{\Psi}^\dagger(\vb{r}')
    \begin{pmatrix}
        \xi(\vb{r})\delta(\vb{r}-\vb{r}')\hat{1} & \hat{\Delta}(\vb{r},\vb{r}')              \\
        \hat{\Delta}^\dagger(\vb{r},\vb{r}')     & -\xi(\vb{r})\delta(\vb{r}-\vb{r}')\hat{1}
    \end{pmatrix}
    \widehat{\Psi}(\vb{r})                              \\
     & \equiv \frac{1}{2}\iint\dd[3]\vb{r}\dd[3]\vb{r}'
    \widehat{\Psi}^\dagger(\vb{r}')
    \widehat{\tau}_3\left(\xi(\vb{r})\delta(\vb{r}-\vb{r}') + \widehat{\Delta}(\vb{r},\vb{r}')\right)
    \widehat{\Psi}(\vb{r}).
\end{align}
where
\begin{align}
    \widehat{\Delta}(\vb{r},\vb{r}') =
    \begin{pmatrix}
        0                                     & \hat{\Delta}(\vb{r},\vb{r}') \\
        -\hat{\Delta}^\dagger(\vb{r},\vb{r}') & 0
    \end{pmatrix}
\end{align}
When the system is homogeneous, the two-point gap function only depends on the
relative coordinate, i.e., $\hat{\Delta}(\vb{r},\vb{r}') = \hat{\Delta}(\vb{r}-\vb{r}')$.
In this case, we can transform to momentum space, and the Hamiltonian becomes
\begin{equation}
    H = \frac{1}{2}\sum_{\vb{p}}
    \widehat{\Psi}_{\vb{p}}^{\dagger}
    \widehat{\tau}_3(\xi_{\vb{p}} + \widehat{\Delta}(\vb{p}))
    \widehat{\Psi}_{\vb{p}}.
\end{equation}
where
$\widehat{\Psi}_{\vb{p}}^\dagger = (c_{\vb{p}\uparrow}^\dagger, c_{\vb{p}\downarrow}^\dagger, c_{-\vb{p}\uparrow}, c_{-\vb{p}\downarrow})$.

\section{Finite temperature}
The imaginary time Green's function is
\begin{align}
    \widehat{G}(x, x') = -\ev{T_\tau \widehat{\Psi}(x) \widehat{\Psi}^\dagger(x')}.
\end{align}
where $x = (\vb{r}, \tau)$. The unperturbed Matsubara Green's function is
\begin{align}
    \widehat{G}_0(\vb{p}, i\epsilon_n)
    = (i\epsilon_n - \widehat{\tau}_3\xi_{\vb{p}})^{-1}
    = -\frac{i\epsilon_n + \widehat{\tau}_3\xi_{\vb{p}}}{\epsilon_n^2 + \xi_{\vb{p}}^2}.
\end{align}
The Gorkov Green's function satisfies
\begin{align}
    \widehat{G}^{-1}(\vb{p}, i\epsilon_n)
    = \widehat{G}_0^{-1}(\vb{p}, i\epsilon_n) - \widehat{\tau}_3\widehat{\Delta}(\vb{p})
    = i\epsilon_n - \widehat{\tau}_3\xi_{\vb{p}} - \widehat{\tau}_3\widehat{\Delta}(\vb{p}).
\end{align}
which gives
\begin{align}
    \widehat{G}(\vb{p}, i\epsilon_n)
    = - \frac{i\epsilon_n + \widehat{\tau}_3\xi_{\vb{p}} + \widehat{\tau}_3\widehat{\Delta}(\vb{p})}
    {\epsilon_n^2 + \xi_{\vb{p}}^2 + |\Delta|^2}
\end{align}
The quasiclassical Green's function is
\begin{align}
    \widehat{g}(\vb{p}, i\epsilon_n)
    = \int\dd\xi_p \widehat{G}(\vb{p}, i\epsilon_n) \widehat{\tau}_3
    = -\pi \frac{i\epsilon_n\widehat{\tau}_3 - \widehat{\Delta}(\vb{p})}
    {\sqrt{|\Delta|^2 + \epsilon_n^2}}
    = \frac{\pi}{\sqrt{|\Delta|^2 + \epsilon_n^2}}
    \begin{pmatrix}
        -i\epsilon_n          & \hat{\Delta} \\
        -\hat{\Delta}^\dagger & i\epsilon_n
    \end{pmatrix}
    \equiv
    \begin{pmatrix}
        \hat{g}          & \hat{f}  \\
        -\hat{f}^\dagger & -\hat{g}
    \end{pmatrix}
\end{align}
and it satisfies the normalization condition
\begin{align}
    \widehat{g}(\vb{p}, i\epsilon_n)^2 = -\pi^2\widehat{1}.
\end{align}
The bulk quasiclassical Green's function satisfies
\begin{align}
    \comm{i\epsilon_n\widehat{\tau}_3 - \widehat{\Delta}}{\widehat{g}} = 0
\end{align}
But when near the boundary or impurities, the
quasiclassical Green's function $\widehat{g}(\vb{r},\vb{p};\epsilon)$
and the gap function $\widehat{\Delta}(\vb{r}, \vb{p})$ are
inhomogeneous, and are described by the
Eilenberger equation and gap equation.
\begin{equation}
    \comm{\epsilon\widehat{\tau}_3 - \widehat{\Delta}(\vb{r}, \vb{p})}
    {\widehat{g}(\vb{r},\vb{p};\epsilon)} +
    i\hbar\vb{v}_p \cdot \nabla\widehat{g}(\vb{r},\vb{p};\epsilon)
    = 0
\end{equation}
and
\begin{align}
    \hat{\Delta}(\vb{r}, \vb{p})
    = \ev{v(\vb{p}, \vb{p}')T\sum_{\epsilon_n=-\omega_c}^{\omega_c}
        \hat{f}(\vb{r}, \vb{p}', \epsilon_n)}_{p'}
\end{align}

\section{Numerical solution}
The $(1,1)$ and $(1,2)$ elements of the Eilenberger equation are
\begin{align}
    i\epsilon_n \hat{g} + \hat{\Delta} \hat{f}^\dagger + i\hbar \vb{v}_p \cdot \grad \hat{g} = 0 \\
    i\epsilon_n \hat{f} + \hat{\Delta} \hat{g} + i\hbar \vb{v}_p \cdot \grad \hat{f} = 0
\end{align}
Note that $\hat{\Delta}$ and $\hat{f}$ has the same spin structure, so we can
substitute in $\hat{g}$ and $\hat{f}$, divide by $T_c$, and we get
\begin{align}
    i (2n+1)\pi t g(\vb{r}, \vu{p}, n) + \hat{\delta}(\vb{r}, \vu{p})
    f^*(\vb{r}, \vu{p}, n) + i(\vu{p}_x \partial_x
    + \vu{p}_y \partial_y) g(\vb{r}, \vu{p}, n) = 0 \\
    i (2n+1)\pi t f(\vb{r}, \vu{p}, n) + \hat{\delta}(\vb{r}, \vu{p})
    g(\vb{r}, \vu{p}, n) + i(\vu{p}_x \partial_x
    + \vu{p}_y \partial_y) f(\vb{r}, \vu{p}, n) = 0
\end{align}
where $t = T/T_c$, $\delta = \Delta/T_c$, $\xi = \hbar v_f/T_c$ and
$\va{\partial} = \xi\grad$.

For the gap equation, we ...
\begin{align}
    \delta =
\end{align}

We first assume an initial guess for the gap function $\delta(\vb{r}, \vu{p})$.
For each trajectory $\vu{p}$ along which the quasiparticle propagates, we can
solve the Eilenberger equation to get the Green's function $g(\vb{r}, \vu{p}, n)$
and the anomalous Green's function $f(\vb{r}, \vu{p}, n)$. Then we iterate to
update the gap function $\delta(\vb{r}, \vu{p})$ by solving the gap equation.
Repeat until convergence.

\section{$^3$He-A}
For $^3$He-A, the spin structure is $\vu{d}\vdot(i\va{\sigma}\sigma_y) = \sigma_x$.
In general, we have
\begin{align}
    \hat{\Delta}(\vb{r}, \vb{p})
    = \hat{\sigma}_x (\Delta_1(\vb{r}, \vb{p}) + i\Delta_2(\vb{r}, \vb{p}))
\end{align}
in bulk we have $\Delta_1(\vb{p}) = \Delta p_x$ and $\Delta_2(\vb{p}) = \Delta p_y$.
We can also write the anomalous Green's function as
\begin{align}
    \hat{f}(\vb{r}, \vb{p}, \epsilon_n)
    = \hat{\sigma}_x \bigg(f_1(\vb{r}, \vb{p}, \epsilon_n) + i f_2(\vb{r}, \vb{p}, \epsilon_n) \bigg)
\end{align}

\section{Edge gap profile}
We assume translational invariance along the edge, i.e. the $y$ direction.
We choose the gap to be real along the $x$ direction. We assume the momentum
dependence of the gap is given by
\begin{align}
    \hat{\Delta}(x, \vb{p})
    = \hat{\sigma}_x (\Delta_1(x)p_x + i\Delta_2(x)p_y)
\end{align}

\end{document}