\documentclass[aps,prl,preprint]{revtex4-2}
\usepackage{amsmath}
\usepackage{physics}

\begin{document}

\title{Title of Your Paper}
\author{Your Name}
\affiliation{Your Institution}
\date{\today}

\begin{abstract}
    Your abstract goes here.
\end{abstract}

% \maketitle
\section{Mean-field theory}
The mean-field Nambu-Gor'kov Hamiltonian is
\begin{align}
    H
     & = \frac{1}{2}\iint\dd[3]\vb{r}\dd[3]\vb{r}'
    \widehat{\Psi}^\dagger(\vb{r}')
    \begin{pmatrix}
        \xi(\vb{r})\delta(\vb{r}-\vb{r}')\hat{1} & \hat{\Delta}(\vb{r},\vb{r}')              \\
        \hat{\Delta}^\dagger(\vb{r},\vb{r}')     & -\xi(\vb{r})\delta(\vb{r}-\vb{r}')\hat{1}
    \end{pmatrix}
    \widehat{\Psi}(\vb{r})                              \\
     & \equiv \frac{1}{2}\iint\dd[3]\vb{r}\dd[3]\vb{r}'
    \widehat{\Psi}^\dagger(\vb{r}')
    \widehat{\tau}_3\left(\xi(\vb{r})\delta(\vb{r}-\vb{r}') + \widehat{\Delta}(\vb{r},\vb{r}')\right)
    \widehat{\Psi}(\vb{r}).
\end{align}
where
\begin{align}
    \widehat{\Delta}(\vb{r},\vb{r}') =
    \begin{pmatrix}
        0                                     & \hat{\Delta}(\vb{r},\vb{r}') \\
        -\hat{\Delta}^\dagger(\vb{r},\vb{r}') & 0
    \end{pmatrix}
\end{align}
When the system is homogeneous, the two-point gap function only depends on the
relative coordinate, i.e., $\hat{\Delta}(\vb{r},\vb{r}') = \hat{\Delta}(\vb{r}-\vb{r}')$.
In this case, we can transform to momentum space, and the Hamiltonian becomes
\begin{equation}
    H = \frac{1}{2}\sum_{\vb{p}}
    \widehat{\Psi}_{\vb{p}}^{\dagger}
    \widehat{\tau}_3(\xi_{\vb{p}} + \widehat{\Delta}(\vb{p}))
    \widehat{\Psi}_{\vb{p}}.
\end{equation}
where
$\widehat{\Psi}_{\vb{p}}^\dagger = (c_{\vb{p}\uparrow}^\dagger, c_{\vb{p}\downarrow}^\dagger, c_{-\vb{p}\uparrow}, c_{-\vb{p}\downarrow})$.

\section{Finite temperature}
The imaginary time Green's function is
\begin{align}
    \widehat{G}(\vb{r}\tau, \vb{r}'\tau')
    = \widehat{G}(\vb{r}, \vb{r}';\tau-\tau')
    = -\ev{T_\tau \widehat{\Psi}(\vb{r}, \tau) \widehat{\Psi}^\dagger(\vb{r}', \tau')}
    \equiv
    \begin{pmatrix}
        \hat{G}             & \hat{F}             \\
        \hat{\underline{F}} & \hat{\underline{G}}
    \end{pmatrix}
\end{align}
The unperturbed Matsubara Green's function is
\begin{align}
    \widehat{G}_0(\vb{p}, i\epsilon_n)
    = (i\epsilon_n - \widehat{\tau}_3\xi_{\vb{p}})^{-1}
    = -\frac{i\epsilon_n + \widehat{\tau}_3\xi_{\vb{p}}}{\epsilon_n^2 + \xi_{\vb{p}}^2}.
\end{align}
In bulk, the Gorkov Green's function satisfies
\begin{align}
    \widehat{G}^{-1}(\vb{p}, i\epsilon_n)
    = \widehat{G}_0^{-1}(\vb{p}, i\epsilon_n) - \widehat{\tau}_3\widehat{\Delta}(\vb{p})
    = i\epsilon_n - \widehat{\tau}_3\xi_{\vb{p}} - \widehat{\tau}_3\widehat{\Delta}(\vb{p}).
\end{align}
which gives
\begin{align}
    \widehat{G}(\vb{p}, i\epsilon_n)
    = - \frac{i\epsilon_n + \widehat{\tau}_3\xi_{\vb{p}} + \widehat{\tau}_3\widehat{\Delta}(\vb{p})}
    {\epsilon_n^2 + \xi_{\vb{p}}^2 + |\Delta|^2}
\end{align}
The bulk quasiclassical Green's function is
\begin{align}
    \widehat{g}(\vb{p}, i\epsilon_n)
    = \int\dd\xi_p \widehat{G}(\vb{p}, i\epsilon_n) \widehat{\tau}_3
    = -\pi \frac{i\epsilon_n\widehat{\tau}_3 - \widehat{\Delta}(\vb{p})}
    {\sqrt{|\Delta|^2 + \epsilon_n^2}}
    = \frac{\pi}{\sqrt{|\Delta|^2 + \epsilon_n^2}}
    \begin{pmatrix}
        -i\epsilon_n          & \hat{\Delta} \\
        -\hat{\Delta}^\dagger & i\epsilon_n
    \end{pmatrix}
    \equiv
    \begin{pmatrix}
        \hat{g}             & \hat{f}             \\
        \hat{\underline{f}} & \hat{\underline{g}}
    \end{pmatrix}
\end{align}
and it satisfies the normalization condition
\begin{align}
    \widehat{g}(\vb{p}, i\epsilon_n)^2 = -\pi^2\widehat{1}.
\end{align}
The bulk quasiclassical Green's function satisfies
\begin{align}
    \comm{i\epsilon_n\widehat{\tau}_3 - \widehat{\Delta}}{\widehat{g}} = 0
\end{align}
But when near the boundary or impurities, the
quasiclassical Green's function $\widehat{g}(\vb{R},\vb{p};\epsilon)$
and the gap function $\widehat{\Delta}(\vb{R}, \vb{p})$ are
inhomogeneous, and are described by the
Eilenberger equation and gap equation.
\begin{equation}
    \comm{\epsilon\widehat{\tau}_3 - \widehat{\Delta}(\vb{R}, \vb{p})}
    {\widehat{g}(\vb{R},\vb{p};\epsilon)} +
    i\hbar\vb{v}_p \cdot \nabla\widehat{g}(\vb{R},\vb{p};\epsilon)
    = 0
\end{equation}
and
\begin{align}\label{gap_eq}
    \hat{\Delta}(\vb{R}, \vb{p})
    = T\sum_{\epsilon_n=-\omega_c}^{\omega_c}\ev{v(\vb{p}, \vb{p}')
        \hat{f}(\vb{R}, \vb{p}', \epsilon_n)}_{p'}
\end{align}
Here $\vb{R} = \frac{1}{2}(\vb{r} + \vb{r}')$ is the center of the pair.
Now we neglect the spin structure, i.e. discard the little hats.
In bulk, we have
\begin{align}\label{gap_eq_bulk}
    \Delta(\vb{p}) = T\sum_{\epsilon_n=-\omega_c}^{\omega_c}
    \ev{v(\vb{p}, \vb{p}')
        \frac{\pi\Delta(\vb{p}')}{\sqrt{|\Delta|^2 + \epsilon_n^2}}}_{p'}
\end{align}
Move everything to the right hand side and subtract Eq. (\ref{gap_eq}) from it, we get
\begin{align}
    0 = T\sum_{\epsilon_n=-\infty}^{\infty}
    \left[
        \ev{v(\vb{p}, \vb{p}')\frac{f(\vb{R}, \vb{p}', \epsilon_n)}{\Delta(\vb{R}, \vb{p})}}_{p'}
        -
        \ev{v(\vb{p}'', \vb{p}')\frac{\pi}{\sqrt{|\Delta|^2 + \epsilon_n^2}}\frac{\Delta(\vb{p}')}{\Delta(\vb{p}'')}}_{p'}
        \right]
\end{align}
which gives
\begin{align}
    \Delta(\vb{R}, \vb{p}) =
    \frac{T\sum_{\epsilon_n}^{\infty}\ev{v(\vb{p}, \vb{p}')\hat{f}(\vb{R}, \vb{p}', \epsilon_n)}_{p'}}
    {T\sum_{\epsilon_n}^{\infty}\ev{v(\vb{p}'', \vb{p}')\frac{\pi}{\sqrt{|\Delta|^2 + \epsilon_n^2}}\frac{\Delta(\vb{p}')}{\Delta(\vb{p}'')}}_{p'}}
\end{align}
We have the Digamma function
\begin{align}
    K(T) = T\sum_{\epsilon_n=-\omega_c}^{\omega_c}
    \frac{\pi}{|\epsilon_n|}\approx\ln{\left(1.13\frac{\omega_c}{T}\right)}
\end{align}
In bulk at $T_c$, we have
\begin{align}
    \Delta(\vb{p}) = K(T_c)
    \ev{v(\vb{p}, \vb{p}')\Delta(\vb{p}')}_{p'}
\end{align}
and we have
\begin{align}
    K(T_c) - K(T) = \ln{\left(T/T_c\right)}
    =\frac{\Delta(\vb{p}'')}{\ev{v(\vb{p}'', \vb{p}')\Delta(\vb{p}')}_{p'}}\eval_{T_c}
    - T\sum_{\epsilon_n=-\omega_c}^{\omega_c}\frac{\pi}{|\epsilon_n|}
\end{align}
Substitute in Eq. (\ref{gap_eq}), we get another equation to determine the gap inhomogeneity:
\begin{align}
    \ln{\frac{T}{T_c}} = T\sum_{\epsilon_n=-\infty}^{\infty}
    \left[
        \frac{\ev{v(\vb{p}, \vb{p}')f(\vb{R}, \vb{p}', \epsilon_n)}_{p'} / \Delta(\vb{R}, \vb{p})}
        {\ev{v(\vb{p}'', \vb{p}')\Delta(\vb{p}')}_{p'}/\Delta(\vb{p}'')\eval_{T_c}}
        - \frac{\pi}{|\epsilon_n|}
        \right]
\end{align}
If you want temperature dependence of bulk gap, you can substitute in Eq. (\ref{gap_eq_bulk}) to get
\begin{align}
    \ln{\frac{T}{T_c}} = T\sum_{\epsilon_n=-\infty}^{\infty}
    \left[
        \frac{\pi\ev{v(\vb{p}, \vb{p}')\Delta(\vb{p}') / \Delta(\vb{p}) \sqrt{|\Delta|^2 + \epsilon_n^2}}_{p'}}
        {\ev{v(\vb{p}'', \vb{p}')\Delta(\vb{p}')}_{p'}/\Delta(\vb{p}'')\eval_{T_c}}
        - \frac{\pi}{|\epsilon_n|}
        \right]
\end{align}

\section{Riccati Parameterizations}
The Riccati parameterization of the Green's function is
\begin{align}
    \widehat{g}
    = -i\pi\widehat{N}
    \begin{pmatrix}
        1+\hat{a}\hat{\underline{a}} & 2\hat{a}                      \\
        -2\hat{\underline{a}}        & -1-\hat{\underline{a}}\hat{a}
    \end{pmatrix}
\end{align}
where $\widehat{N}$ is
\begin{align}
    \widehat{N} =
    \begin{pmatrix}
        (1-\hat{a}\hat{\underline{a}})^{-1} & 0                                   \\
        0                                   & (1-\hat{\underline{a}}\hat{a})^{-1}
    \end{pmatrix}
\end{align}
The inverse is
\begin{align}
    \hat{a} = (\hat{g}-i\pi)^{-1}\hat{f} \\
    \hat{\underline{a}} = (\hat{\underline{g}}+i\pi)^{-1}\hat{\underline{f}}
\end{align}
Substitute in the bulk Green's function to get the bulk Ricatti amplitude:
\begin{align}
    \hat{a} = \frac{i\hat{\Delta}}{\epsilon_n + \sqrt{|\Delta|^2 + \epsilon_n^2}} \\
    \hat{\underline{a}} = \frac{i\hat{\Delta}^\dagger}{\epsilon_n + \sqrt{|\Delta|^2 + \epsilon_n^2}}
\end{align}
The matrix Ricatti equation is
\begin{align}
    i\hbar\vb{v}_p \vdot \grad{\hat{a}} + 2\epsilon\hat{a}
    + \hat{a}\hat{\Delta}^\dagger\hat{a} + \hat{\Delta} = 0 \\
    i\hbar\vb{v}_p \vdot \grad{\hat{\underline{a}}} - 2\epsilon\hat{\underline{a}}
    - \hat{\underline{a}}\hat{\Delta}\hat{\underline{a}} - \hat{\Delta}^\dagger = 0
\end{align}

\section{$^3$He-A}
For $^3$He-A, the spin structure is $\vu{d}\vdot(i\va{\sigma}\hat{\sigma}_y) = \hat{\sigma}_x$,
and we have two components of the gap function,
\begin{align}
    \hat{\Delta}(\vb{R}, \vb{p})
    = \hat{\sigma}_x (\Delta_1(\vb{R})p_x + \Delta_2(\vb{R})p_y)
\end{align}
In bulk we have
$\hat{\Delta}(\vb{R}, \vb{p}) = \hat{\sigma}_x \Delta(p_x + ip_y) = \hat{\sigma}_x \Delta e^{i\phi_p}$.
The interaction is $v(\vb{p}, \vb{p}') = 3v_0\vu{p}\vdot\vu{p}'$, and the
gap-equation reduces to
\begin{align}
    \ln{\frac{T}{T_c}} = T\sum_{\epsilon_n=-\infty}^{\infty}
    \left[
        \frac{2\ev{\vb{p}_{x,y}\vdot\vb{p}'f(\vb{R}, \vb{p}', \epsilon_n)}_{p'}}
        {\Delta_{1,2}(\vb{R})}
        - \frac{\pi}{|\epsilon_n|}
        \right]
\end{align}
The 2d bulk gap equation ($|\Delta|^2$ does not depend on $\vb{p}$) is
\begin{align}
    \ln{\frac{T}{T_c}} = T\sum_{\epsilon_n=-\infty}^{\infty}
    \left[
        \frac{\pi}{\sqrt{|\Delta|^2 + \epsilon_n^2}}
        - \frac{\pi}{|\epsilon_n|}
        \right]
\end{align}

\section{s-wave pairing}
For s-wave pairing, the spin structure is $i\hat{\sigma}_y$, and we have
\begin{align}
    \hat{\Delta}(\vb{R}, \vb{p})
    = i\hat{\sigma}_y \Delta(\vb{R})
\end{align}
In bulk we have $\hat{\Delta}(\vb{R}, \vb{p}) = i\hat{\sigma}_y \Delta$.
The interaction is $v(\vb{p}, \vb{p}') = v_s$, and the gap-equation reduces to
\begin{align}
    \ln{\frac{T}{T_c}} = T\sum_{\epsilon_n=-\infty}^{\infty}
    \left[
        \frac{f(\vb{R}, \epsilon_n)}{\Delta(\vb{R})}
        - \frac{\pi}{|\epsilon_n|}
        \right]
\end{align}

\section{$^3$He-A edge gap profile}
We assume translational invariance along the edge, i.e. the $y$ direction.
We also assume the gap to be real(imaginary) along the $p_x(p_y)$ direction.
\begin{align}
    \hat{\Delta}(x, \vb{p})
    = \hat{\sigma}_x (\Delta_1(x)p_x + i\Delta_2(x)p_y)
\end{align}
Here $\Delta_1(x)$ and $\Delta_2(x)$ are real functions, and we have
\begin{align}
    \widehat{\Delta}(x, \vb{p})
     & =
    \begin{pmatrix}
        0                                                  & \hat{\sigma}_x (\Delta_1(x)p_x + i\Delta_2(x)p_y) \\
        -\hat{\sigma}_x (\Delta_1(x)p_x - i\Delta_2(x)p_y) & 0
    \end{pmatrix} \\
     & = i\hat{\sigma}_x(\Delta_2p_y\widehat{\tau}_1 + \Delta_1p_x\widehat{\tau}_2)
\end{align}
We can always write the anomalous Green's function as
\begin{align}
    \hat{f}(\vb{R}, \vb{p}, \epsilon_n)
    = \hat{\sigma}_x \bigg(f_1(\vb{R}, \vb{p}, \epsilon_n) + if_2(\vb{R}, \vb{p}, \epsilon_n) \bigg)
\end{align}
where $f_{1,2}$ are real functions, and we have
\begin{align}
    \widehat{g}
    = \hat{g}\widehat{\tau}_3 + i\hat{\sigma}_x(f_2\widehat{\tau}_1 + f_1\widehat{\tau}_2)
\end{align}
In this case, we can even write the gap equation as
\begin{align}
    \Delta_{1,2}(x) = T\sum_{\epsilon_n=-\omega_c}^{\omega_c}\ev{v(\vb{p}_{x,y}, \vb{p}')
        f_{1,2}(x, \vb{p}', \epsilon_n)}_{p'}
\end{align}
which becomes
\begin{align}
    \ln{\frac{T}{T_c}} = T\sum_{\epsilon_n=-\infty}^{\infty}
    \left[
    \frac{2\ev{\vb{p}_{x,y}\vdot\vb{p}'f_{1,2}(x, \vb{p}', \epsilon_n)}_{p'}}
    {\Delta_{1,2}(x)}
    - \frac{\pi}{|\epsilon_n|}
    \right]
\end{align}

\subsection{Numerical solution}
The iteration of the gap function is
\begin{align}
    \delta_{1,2}(x) =
    \frac{\sum_n \ev{\vb{p}_{x,y}\vdot\vb{p}'f_{1,2}(x, \vb{p}', n)}_{p'}}
    {\sum_n \frac{\pi}{\sqrt{1 + e_n^2/\delta(t)^2}}\frac{\pi}{2}}
\end{align}
or equivalently
\begin{align}
    \delta_{1,2}(x)\delta(T) =
    \frac{t\sum_n2\ev{\vb{p}_{x,y}\vdot\vb{p}'f_{1,2}(x, \vb{p}', n)}_{p'}}
    {\ln{t} + \sum_n 1/|2n+1|}
\end{align}
where $t = T/T_c$, $\delta(\vb{p},x) = \Delta(\vb{p},x)/|\Delta(T)|$, and $\delta(T) = |\Delta(T)|/T_c$.
The Ricatti equation is
\begin{align}
    ip_x\partial_x a(\vb{p},x) + 2ea(\vb{p},x) + a(\vb{p},x)\delta^*(\vb{p},x) a(\vb{p},x) + \delta(\vb{p},x) = 0 \\
    ip_x\partial_x \underline{a}(\vb{p},x) - 2e \underline{a}(\vb{p},x) - \underline{a}(\vb{p},x)\delta(\vb{p},x)\underline{a}(\vb{p},x) - \delta^*(\vb{p},x) = 0
\end{align}
where $e = \epsilon/|\Delta(T)|$,
$\xi = \hbar v_f/|\Delta(T)|$ and $\va{\partial} = \xi\grad$.
For Matsubara frequency, we replace $e -> i(2n+1)\pi t/\delta(T)$.
We can get $g$ and $f_{1,2}$ from
\begin{align}
    g = -i\pi\frac{1+a\underline{a}}{1-a\underline{a}} \\
    f = f_1 + if_2 = -i\pi\frac{2a}{1-a\underline{a}}
\end{align}
The local current density is
\begin{align}
    \vb{j}(\vb{R}) = 2N_fT\sum_n\ev{\vb{v}_p g(\vb{R}, \vb{p}, \epsilon_n)}_{p}
\end{align}

We first assume an initial guess for the gap function $\delta(\vb{r}, \vu{p})$.
For each trajectory $\vu{p}$ along which the quasiparticle propagates, we can
solve the Eilenberger equation to get the Green's function $g(\vb{r}, \vu{p}, n)$
and the anomalous Green's function $f(\vb{r}, \vu{p}, n)$. Then we iterate to
update the gap function $\delta(\vb{r}, \vu{p})$ by solving the gap equation.
Repeat until convergence.

\end{document}